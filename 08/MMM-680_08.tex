%%%%%%%%%%%%%%%%%%%%%%%%%%%%%%%%%%%%%%%%%%%%%%%%%%%%%%%%%%%%%
%   TEMPLATE TO RUN A CLASS.TEX FILE WITHOUT THE REST OF    %
%   CLASS FILES                                             %
%%%%%%%%%%%%%%%%%%%%%%%%%%%%%%%%%%%%%%%%%%%%%%%%%%%%%%%%%%%%%

\documentclass[aspectratio=169]{beamer} % Puedes ajustar el aspectratio a 43 si lo prefieres
\usepackage[utf8]{inputenc}
\usepackage[spanish]{babel}
\usepackage{minted} % Para resaltar código y comandos
\usetheme{Madrid} % Un tema común y limpio, puedes elegir otro

% Configuración para minted
\setminted{breaklines, breakanywhere, fontsize=\small}

\title{Introducción a Docker Containers}
\date{\today}

\begin{document}

\begin{frame}
    \titlepage
\end{frame}

\begin{frame}{¿Qué es Docker?}
    \begin{itemize}
        \item Docker es un proyecto de código abierto pero de una compañia privada que \textbf{automatiza el despliegue de aplicaciones de software dentro de contenedores}.
        \item Proporciona una capa adicional de abstracción de \textbf{virtualización a nivel de SO} en Linux.
        \item En palabras más sencillas, Docker es una herramienta que permite a desarrolladores, administradores de sistemas, etc., \textbf{desplegar fácilmente sus aplicaciones en un entorno aislado (llamado \textit{contenedores})} para que se ejecuten en el sistema operativo anfitrión, es decir, Linux.
        \item El beneficio clave de Docker es que permite a los usuarios \textbf{empaquetar una aplicación con todas sus dependencias en una unidad estandarizada} para el desarrollo de software.
    \end{itemize}
\end{frame}

\begin{frame}{¿Qué son los Contenedores? (\textit{Containers})}
    \begin{itemize}
        \item Antes se usaban mucho las Máquinas Virtuales.
        \item Las VMs ejecutan aplicaciones dentro de un \textbf{sistema operativo invitado} que se ejecuta en hardware virtualizado, lo que conlleva una sobrecarga computacional sustancial.
        \item Los \textbf{contenedores} son diferentes: usan los mecanismos de bajo nivel del sistema operativo anfitrión para proporcionar la mayor parte del aislamiento de las máquinas virtuales, pero \textbf{con consumen menos recursos}.
        \item Ofrecen un \textbf{mecanismo de empaquetado lógico} en el que las aplicaciones pueden ser abstraídas del entorno en el que realmente se ejecutan.
        \item Este desacoplamiento permite que las \textbf{aplicaciones basadas en contenedores sean portatiles}.
        \item Esto permite crear \textbf{entornos predecibles que están aislados} del resto de las aplicaciones y pueden ejecutarse en cualquier lugar.
        \item También ofrecen control más granular de los recursos.
    \end{itemize}
\end{frame}

\begin{frame}{Terminología Básica de Docker}
    La jerga específica de Docker puede ser confusa al principio. Aquí algunos términos clave:
    \begin{itemize}
        \item \textbf{Imágenes (\textit{Images})}: Son los planos de nuestra aplicación que forman la base de los contenedores. Se descargan del Docker registry (ej. Docker Hub) usando \texttt{docker pull}.
        \item \textbf{Contenedores (\textit{Containers})}: Se crean a partir de imágenes Docker y ejecutan la aplicación real. Se inician usando \texttt{docker run}.
        \item \textbf{Docker Daemon}: El servicio en segundo plano que se ejecuta en el anfitrión y gestiona la construcción, ejecución y distribución de contenedores Docker. Es el proceso con el que interactúan los clientes.
        \item \textbf{Docker Client}: La herramienta de línea de comandos que permite al usuario interactuar con el daemon. También pueden existir otros clientes con GUI.
        \item \textbf{Docker Hub}: Un registro de imágenes Docker. Puedes pensarlo como un directorio de todas las imágenes Docker disponibles.
    \end{itemize}
\end{frame}

\begin{frame}[containsverbatim]{Instalación de Docker}
    \begin{itemize}
        \item Docker (la compañia) ha invertido significativamente en mejorar la experiencia de incorporación para sus usuarios en sistemas operativos como Mac, Linux y Windows, haciendo que su ejecución sea muy sencilla.
        \item Después de la instalación, puedes verificar que Docker funciona correctamente ejecutando un comando simple:
        \begin{minted}{bash}
$ docker run hello-world
        \end{minted}
        \item Esto debería mostrar un mensaje de ``Hello from Docker", confirmando que tu instalación es correcta.
    \end{itemize}
\end{frame}

\begin{frame}[containsverbatim]{Uso Básico: Imágenes}
    \begin{itemize}
        \item Para obtener una nueva imagen Docker, puedes descargarla de un registro (como Docker Hub) o crear la tuya propia.
        \item Existen decenas de miles de imágenes disponibles en Docker Hub.
        \item  Use el comando \texttt{docker pull} para descargar una imagen de un registro a tu sistema.
        \item Para ver una lista de todas las imágenes disponibles localmente en tu sistema, usa el comando \texttt{docker images}.
    \end{itemize}
    \vspace{0.5cm}
    \textbf{Ejemplo: Descargar la imagen \texttt{busybox}}
    \begin{minted}{bash}
$ docker pull busybox
# Dependiendo de tu instalación, podrías necesitar 'sudo' en Linux.
    \end{minted}
\end{frame}
\begin{frame}[containsverbatim]{Uso Básico: Imágenes}
    \textbf{Ejemplo: Listar imágenes locales}
    \begin{minted}{bash}
$ docker images
REPOSITORY    TAG      IMAGE ID       CREATED        VIRTUAL SIZE
busybox       latest   c51f86c28340   4 weeks ago    1.109 MB
    \end{minted}
    El \texttt{TAG} se refiere a una instantánea particular de la imagen y el \texttt{IMAGE ID} es su identificador único.
\end{frame}

\begin{frame}[containsverbatim]{Uso Básico: Contenedores (Parte 1)}
    \begin{itemize}
        \item Para ejecutar un contenedor basado en una imagen, se utiliza el comando \texttt{docker run}.
    \end{itemize}
    \vspace{0.5cm}
    \textbf{Ejemplo: Ejecutar \texttt{busybox} sin comando explícito}
    \begin{minted}{bash}
$ docker run busybox
$ # ¡No pasó nada visible! El contenedor se inició, ejecutó un comando vacío y salió.
    \end{minted}
    \vspace{0.5cm}
    \textbf{Ejemplo: Ejecutar \texttt{busybox} con un comando}
    \begin{minted}{bash}
$ docker run busybox echo "hello from busybox"
hello from busybox
    \end{minted}
    \begin{itemize}
        \item El cliente de Docker ejecutó diligentemente el comando \texttt{echo} en el contenedor \texttt{busybox} y luego salió.
        \item Note que los contenedores son mucho más rápidos que las maquinas virtuales.
    \end{itemize}
\end{frame}

\begin{frame}[containsverbatim]{Uso Básico: Contenedores (Parte 2)}
    \begin{itemize}
        \item El comando \texttt{docker ps} muestra todos los contenedores que se están ejecutando actualmente.
        \item Para ver una lista de todos los contenedores que se han ejecutado (incluyendo los que ya salieron), usa \texttt{docker ps -a}.
    \end{itemize}
    \vspace{0.5cm}
    \textbf{Ejemplo: \texttt{docker ps -a}}
    \begin{minted}{bash}
$ docker ps -a
CONTAINER ID   IMAGE         COMMAND     CREATED         STATUS                 PORTS    NAMES
305297d7a235   busybox       "uptime"    11 minutes ago  Exited (0) 11 minutes ago distracted_goldstine
ff0a5c3750b9   busybox       "sh"        12 minutes ago  Exited (0) 12 minutes ago elated_ramanujan
14e5bd11d164   hello-world   "/hello"    2 minutes ago   Exited (0) 2 minutes ago thirsty_euclid
    \end{minted}
\end{frame}
\begin{frame}[containsverbatim]{Uso Básico: Contenedores (Parte 2)}

    \textbf{Ejemplo: Ejecutar un contenedor en modo interactivo}
    \begin{itemize}
        \item Para ejecutar múltiples comandos en un contenedor, puedes usar las banderas \texttt{-it} con \texttt{docker run}.
        \item Esto te conecta a una TTY interactiva en el contenedor.
    \end{itemize}
    \begin{minted}{bash}
$ docker run -it busybox sh
/ # ls
bin dev etc home proc root sys tmp usr var
/ # uptime
05:45:21 up 5:58, 0 users, load average: 0.00, 0.01, 0.04
/ # exit
    \end{minted}
\end{frame}

\begin{frame}[containsverbatim]{Uso Básico: Eliminación de Contenedores e Imágenes}
    \begin{itemize}
        \item Dejar contenedores inactivos consume espacio en disco, por lo que se recomienda limpiarlos una vez que hayas terminado con ellos.
        \item Puedes eliminar contenedores usando \texttt{docker rm} y el ID del contenedor.
    \end{itemize}
    \vspace{0.5cm}
    \textbf{Ejemplo: Eliminar contenedores por ID}
    \begin{minted}{bash}
$ docker rm 305297d7a235 ff0a5c3750b9
305297d7a235
ff0a5c3750b9
    \end{minted}
    \vspace{0.5cm}
    \textbf{Ejemplo: Eliminar todos los contenedores con estado 'exited'}
    \begin{minted}{bash}
        $ docker container prune
    \end{minted}

    \vspace{0.5cm}

    \textbf{Ejemplo: Eliminar imágenes que ya no necesitas}
    \begin{minted}{bash}
$ docker rmi yourusername/catnip
    \end{minted}
\end{frame}

\begin{frame}[containsverbatim]{Ejecutando un Sitio Web Estático con Docker}
Es posible descargar y ejecutar una imagen directamente en una sola vez usando \texttt{docker run}.
    \vspace{0.5cm}
    \textbf{Ejemplo: Ejecutar un sitio web estático (imagen \texttt{prakhar1989/static-site})}
    \begin{itemize}
        \item \texttt{--rm} elimina automáticamente el contenedor al salir.
        \item \texttt{-it} especifica una terminal interactiva para facilitar la detención con Ctrl+C.
    \end{itemize}
    \begin{minted}{bash}
$ docker run --rm -it prakhar1989/static-site
    \end{minted}
    Si la imagen no existe localmente, el cliente la descargará primero y luego la ejecutará.
\end{frame}

\begin{frame}[containsverbatim]{Uso Básico de Contenedores}
    \textbf{Ejemplo: Ejecutar en modo \textbf{desacoplado} y publicar puertos}
    \begin{itemize}
        \item \texttt{-d} desacopla tu terminal.
        \item \texttt{-P} publica todos los puertos expuestos a puertos aleatorios del host.
        \item \texttt{--name} asigna un nombre al contenedor.
    \end{itemize}
    \begin{minted}{bash}
$ docker run -d -P --name static-site prakhar1989/static-site
e61d12292d69556eabe2a44c16cbd54486b2527e2ce4f95438e504afb7b02810
    \end{minted}
    \vspace{0.2cm}
    \textbf{Verificar los puertos publicados:}
    \begin{minted}{bash}
$ docker port static-site
80/tcp -> 0.0.0.0:32769
443/tcp -> 0.0.0.0:32768
    \end{minted}
    Ahora puedes abrir \textbf{\texttt{http://localhost:32769}} en tu navegador.
\end{frame}

\begin{frame}[containsverbatim]{Uso Básico: Contenedores (Parte 2)}
    \textbf{Detener un contenedor desacoplado:}
    \begin{minted}{bash}
$ docker stop static-site
static-site
    \end{minted}
    Desplegar esto en un servidor real solo requeriría instalar Docker y ejecutar el comando anterior.
\end{frame}


\end{document}
